%%%%%%%%%%%%%%%%%%%%%%%%%%%%%%%%%%%%%%%%%
% Compact Academic CV
% LaTeX Template
% Version 2.0 (6/7/2019)
%
% This template originates from:
% https://www.LaTeXTemplates.com
%
% Authors:
% Dario Taraborelli (http://nitens.org/taraborelli/home)
% Vel (vel@LaTeXTemplates.com)
%
% License:
% CC BY-NC-SA 3.0 (http://creativecommons.org/licenses/by-nc-sa/3.0/)
%
%%%%%%%%%%%%%%%%%%%%%%%%%%%%%%%%%%%%%%%%%

%----------------------------------------------------------------------------------------
%	PACKAGES AND OTHER DOCUMENT CONFIGURATIONS
%----------------------------------------------------------------------------------------

\documentclass[11pt]{article} % Default document font size

\usepackage{ifluatex,ifxetex}
\ifluatex
\usepackage{luatexja}
\fi
\ifxetex
\usepackage{xeCJK}
\setCJKmainfont{IPAMincho}
\setCJKsansfont{IPAGothic}
\setCJKmonofont{IPAGothic}
\fi

%%%%%%%%%%%%%%%%%%%%%%%%%%%%%%%%%%%%%%%%%
% Compact Academic CV
% Structural Definitions
% Version 1.0 (6/7/2019)
%
% This template originates from:
% https://www.LaTeXTemplates.com
%
% Authors:
% Dario Taraborelli (http://nitens.org/taraborelli/home)
% Vel (vel@LaTeXTemplates.com)
%
% License:
% CC BY-NC-SA 3.0 (http://creativecommons.org/licenses/by-nc-sa/3.0/)
%
%%%%%%%%%%%%%%%%%%%%%%%%%%%%%%%%%%%%%%%%%

%----------------------------------------------------------------------------------------
%	REQUIRED PACKAGES AND MISC CONFIGURATIONS
%----------------------------------------------------------------------------------------

\usepackage{graphicx} % Required for including images

\setlength{\parindent}{0pt} % Stop paragraph indentation

%----------------------------------------------------------------------------------------
%	MARGINS
%----------------------------------------------------------------------------------------

\usepackage{geometry} % Required for adjusting page dimensions and margins

\geometry{
	paper=a4paper, % Paper size, change to letterpaper for US letter size
	top=3.25cm, % Top margin
	bottom=4cm, % Bottom margin
	left=3.5cm, % Left margin
	right=3.5cm, % Right margin
	headheight=0.75cm, % Header height
	footskip=1cm, % Space from the bottom margin to the baseline of the footer
	headsep=0.75cm, % Space from the top margin to the baseline of the header
	%showframe, % Uncomment to show how the type block is set on the page
}

%----------------------------------------------------------------------------------------
%	FONTS
%----------------------------------------------------------------------------------------

\usepackage[utf8]{inputenc} % Required for inputting international characters
\usepackage[T1]{fontenc} % Output font encoding for international characters

\usepackage[semibold]{ebgaramond} % Use the EB Garamond font with a reduced bold weight

%----------------------------------------------------------------------------------------
%	SECTION STYLING
%----------------------------------------------------------------------------------------

\usepackage{sectsty} % Allows changing the font options for sections in a document

\sectionfont{\fontsize{13.5pt}{18pt}\selectfont} % Set font options for sections
\subsectionfont{\mdseries\scshape\normalsize} % Set font options for subsections
\subsubsectionfont{\mdseries\upshape\bfseries\normalsize} % Set font options for subsubsections

%----------------------------------------------------------------------------------------
%	MARGIN YEARS
%----------------------------------------------------------------------------------------

\usepackage{marginnote} % Required to output text in the margin

\newcommand{\years}[1]{\marginnote{\scriptsize #1}} % New command for adding years to the margin
\renewcommand*{\raggedleftmarginnote}{} % Left-align the years in the margin
\setlength{\marginparsep}{-0pt} % Move the margin content closer to the text
\reversemarginpar % Margin text to be output into the left margin instead of the default right margin

%----------------------------------------------------------------------------------------
%	COLOURS
%----------------------------------------------------------------------------------------

\usepackage[usenames, dvipsnames]{xcolor} % Required for specifying colours by name
\usepackage{color}

\definecolor{color-text}{gray}{0.10}    % light black
\definecolor{color-detail}{gray}{0.40}  % dark gray


%----------------------------------------------------------------------------------------
%	LINKS
%----------------------------------------------------------------------------------------

\usepackage[bookmarks, colorlinks, breaklinks]{hyperref} % Required for links

% Set link colours
\hypersetup{
	linkcolor=blue,
	citecolor=blue,
	filecolor=black,
	urlcolor=MidnightBlue
}

%----------------------------------------------------------------------------------------
%	ENTRIES
%----------------------------------------------------------------------------------------
\newcommand{\minientry}[2]{\noindent\years{#1} #2\smallskip}

\newcommand{\grant}[3]{\noindent\years{#1}\textbf{#2} \newline #3 \vspace*{5pt}}
\newcommand{\publication}[3]{\noindent\years{#1}\textbf{#2} \newline #3 \vspace*{5pt}}

%----------------------------------------------------------------------------------------
%	SECTION FORMATS
% ---------------------------------------------------------------------------------------- % Include the file specifying the document structure and styling
\newcommand{\YusukeIzawa}{\underline{Yusuke Izawa}}
\newcommand{\伊澤侑祐}{\underline{伊澤侑祐}}
\newcommand{\jit}{\textsc{JIT} }

% Set PDF meta-information
\hypersetup{
	pdftitle={Yusuke Izawa - Curriculum vitae},
	pdfauthor={Yusuke Izawa}
}

%----------------------------------------------------------------------------------------

\begin{document}

%----------------------------------------------------------------------------------------
%	CONTACT AND GENERAL INFORMATION
%----------------------------------------------------------------------------------------

{\LARGE\bfseries Yusuke \textsc{Izawa}} % Name
\bigskip\bigskip\medskip % Whitespace

2 Chome-12-1 Ookayama\\
Meguro City, Tokyo 152-8550, Japan\\ % Address

\medskip

Phone: +8190-4027-6104\\ % Phone number
Email: \href{mailto:izawa@prg.is.titech.ac.jp}{izawa@prg.is.titech.ac.jp}\\ % Email address
\textsc{url}: \href{https://3tty0n.github.io}{https://3tty0n.github.io} \\ % Web site

%------------------------------------------------

\section*{Current position}

\emph{PhD Candidate}, Department of Mathematical and Computing Science,
Tokyo Institute of Technology % Current or most recent employment position

%------------------------------------------------

\section*{Areas of specialisation}

Computer Science; Programming Language % Primary areas of research interest

%----------------------------------------------------------------------------------------
%	WORK EXPERIENCE
%----------------------------------------------------------------------------------------

\section*{Work experience}

\minientry{2018-2019}{Recruit Marketing Partners, Inc., Software Engineer, Self-employment.}

\minientry{2018}{Cookpad, Inc., Software Engineer, Internship. (Won \textbf{2nd Place})}

\minientry{2016-2017}{FOLIO, Inc., Software Engineer, Internship.}

\minientry{2016-2017}{KADOKAWA, Inc., Software Engineer, Internship.}

\minientry{2015-2016}{Summaly, Inc., Software Engineer, Internship.}

%----------------------------------------------------------------------------------------
%	EDUCATION
%----------------------------------------------------------------------------------------

\section*{Education}

\minientry{2020}{\textsc{MSc} in Mathematical and Computing Science, Tokyo Institute of Technology.}

\minientry{2018}{\textsc{BSc} in Mathematical and Computing Science, Tokyo Institute of Technology.}

%----------------------------------------------------------------------------------------
%	GRANTS, HONOURS AND AWARDS
%----------------------------------------------------------------------------------------

\section*{Grants, honours \& awards}

\grant
{2021.4-2023.3}
{Research Fellowship for Young Scientists (JSPS DC2)~~【日本学術振興会特別研究員DC2】.}
{Fellowship from the Japan Society for the Promotion of Science (JSPS), covering
  living expenses. Research expenses covered by KAKENHI (科研費).}

% \grant
% {2020.11-2023.3}
% {JST ACT-X.}
% {Research expenses covered by Japan Science and Technology Agency (JST).}

\grant
{2020.4-2021.3}
{Tokyo Tech Tsubame Scholarship for Doctoral Students~~【東京工業大学 つばめ博士学生
  奨学金】.}
{Tokyo Tech's scholarship for doctoral students, covering living expenses.}

\grant
{2019.4}
{2nd Place, Graduate Category, ACM Student Research Competition, \textit{Association for Computing Machinery.}
  [$\star$]}
{}

\grant
{2019.4}
{Travel Grants by Information Science Incentive Fund~~【情報科学奨励基金】.}
{Covered by dept. of mathematical and computing science, Tokyo Tech.}

% \grant
% {2015}
% {First Semester Tuition Waiver~~~【前期授業料免除】}
% {Waiving the half of tuition of bachelor courses at Tokyo Institute of Technology.}

\grant
{2014.4-2018.3}
{Scholarship by the Showa Scholarship Foundation~~~【昭和奨学会奨学金】.}
{Japanese scholarship by Showa Scholarship Foundation (昭和奨学会), covering
  living expenses.}

%----------------------------------------------------------------------------------------
%	PUBLICATIONS AND TALKS
%----------------------------------------------------------------------------------------

\section*{Publications \& talks}

\subsection*{Peer-reviewed}

\years{2020} Hidehiko Masuhara, Shusuke Takahasi, \YusukeIzawa, and Youyou
Cong. Toward a Multi-Language and Multi-Environment Framework for Live Programming.
Proceedings of the LIVE 2020 Workshop co-located with SPLASH 2020. November 2020. (to
appear)
\medskip

\years{2020} \YusukeIzawa, and Hidehiko Masuhara. Amalgamating Different \jit
Compilations in a Meta-tracing \jit Compiler Framework. Proceedings of the 16th ACM
SIGPLAN International Symposium on Dynamic Languages (DLS'20). November 2020. 15
pages. (to appear)
\medskip

\years{2019} \YusukeIzawa, Hidehiko Masuhara, and Tomoyuki
Aotani. 2019. Extending a meta-tracing compiler to mix method and tracing
compilation. In Proceedings of the Conference Companion of the 3rd International
Conference on Art, Science, and Engineering of Programming (Programming
’19). ACM, New York, NY, USA, Article 5, 1–3. DOI:
\url{https://doi.org/10.1145/3328433.3328439}
\medskip

\years{2019} \YusukeIzawa. BacCaml: the meta-hybrid just-in-time compiler. In
Proceedings of the Conference Companion of the 3rd International Conference on
Art, Science, and Engineering of Programming (Programming ’19). ACM, New York,
NY, USA, Article 32, 1–3. DOI:\url{https://doi.org/10.1145/3328433.3328466}
(\textbf{Awarded} [$\star$])
\medskip

\subsection*{Non-peer-reviewed}

\years{2020} 高橋修祐, \伊澤侑祐, 増原英彦. ライブプログラミング環境は多言語化/多
開発環境化の夢を見るか. The 37th JSSST Anual Conference. Japan Society for
Software Science and Technology (JSSST'20). No. 69. September 2020. Poster
presentation (in Japanese).
\medskip

\years{2020} \YusukeIzawa. Toward Hybrid Compilation in a Practical Meta-tracing
\jit Compiler. PLMW: Programming Language Mentoring Workshop co-located with 41st
ACM SIGPLAN Conference on Programming Language Design and Implementation
(PLDI'20). June 2020. Short talk.
\medskip

\years{2020} \YusukeIzawa, and Hidehiko Masuhara. Making different \jit
compilations dancing to the same tune, acting in the meta-level. The 22nd JSSST
Workshop on Programming and Programming Languages (PPL '20), March 2020. Poster
presentation.
\medskip

\years{2019} \YusukeIzawa, Hidehiko Masuhara, Tomoyuki Aotani, and Youyou
Cong. A stack hybridization for meta-hybrid just-in-time compilation. In Kei
Ito, editor, Proceedings of the 36th JSSST Annual Conference, pages
No. 2–L. Japan Society for Software Science and Technology (JSSST'19), August
2019.
\medskip

% \years{2018} \YusukeIzawa, Hidehiko Masuhara, and Tomoyuki Aotani.  Meta-hybrid
% \jit Compilation Approach to the Path Divergence Problem. Kumiki 6.0. November
% 2018. Oral presentation.
% \medskip

\years{2018} \伊澤侑祐, 増原英彦, 青谷知幸. メタ混合 \jit コンパイラの提案. The
20nd JSSST Workshop on Programming and Programming Languages (PPL '18), March
2018. Poster presentation (in Japanese).
\medskip

%----------------------------------------------------------------------------------------
%	TEACHING
%----------------------------------------------------------------------------------------

\section*{Teaching}

\years{2020 (4Q)} Programming II, Tokyo Institute of Technology, Teaching Assistant

\years{2019 (1Q)} Programming II, Tokyo Institute of Technology, Teaching Assistant

\years{2019 (1Q)} Introduction to Computer Science, Tokyo Institute of Technology, Teaching Assistant

\years{2018 (3Q)} Programming I, Tokyo Institute of Technology, Teaching Assistant

\years{2018 (1Q)} Information Literacy 1, Tokyo Institute of Technology, Teaching Assistant

%------------------------------------------------

\section*{Academic services}

\years{2020} Co-reviewer of Onward! Essays, SPLASH 2020, November 2020.

\years{2020} Candidate of Programming Language Mentoring Workshop, PLDI 2020, June 2020.

\years{2019} Member of Student Volunteer, Programming 2019, April 2019.

\section*{Projects}

\subsection*{BacCaml \, \href{https://github.com/prg-titech/baccaml}{(source code at GitHub)}}

\medskip

Trace-based compilation and method-based compilation are two major compilation
strategies in \jit compilers. In general, the former excels in  compiling
programs with deeper method calls and more dynamic branches, while  the latter
is suitable wide range of programs.

\medskip

This project aims at developing fundamental mechanism for compiling with both
trace-based and method-based strategies. Instead of developing  a compiler for
one particular language, we provide such a mechanism in a meta-compilation
framework, which generates a virtual machine with a \jit compiler from an
interpreter definition of a programming language.

\medskip

We are developing the BacCamel meta-compiler framework as a proof-of-concept,
which is based on the MinCaml compiler.

\medskip

\subsection*{Poly$^2$ Kanon \,
  \href{https://github.com/prg-titech/Kanon}{(source code at GitHub)}}

Kanon is a live programming environment that automatically and instantly
visualizes runtime data-structures while the programmer is editing a
program. Our goal is to have all languages running on top of Kanon and to make
the program run fast.


\vfill % Whitespace before final footer

%----------------------------------------------------------------------------------------
%	FINAL FOOTER
%----------------------------------------------------------------------------------------

% Any final footer text such as a URL to the latest version of this CV, last updated date, compiled in XeTeX, etc
\begin{center}
  \scriptsize
  Last updated: \today~~\raisebox{-0.5pt}{\textbullet}~~
  \href{https://3tty0n.github.io}{https://3tty0n.github.io}
\end{center}

%----------------------------------------------------------------------------------------

\end{document}

%%% Local Variables:
%%% TeX-master: "main"
%%% TeX-engine: luatex
%%% fill-column: 85
%%% End:
